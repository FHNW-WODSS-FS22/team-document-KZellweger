\section{Backend}

\subsection{Aufbau}

Der Aufbau der Serverapplikation lehnt sich am Konzept der Onion-Architecture an.
In Onion Architecture wird die Applikation in Layer aufgeteilt.

\begin{figure}[h]
    \centering
    \begin{minipage}[b]{0.4\textwidth}
        \includegraphics[width=\textwidth]{images/thinktocode-onion}
        \caption{Onion Architecture}
    \end{minipage}\label{fig:figureonion}
\end{figure}

Um zu garantieren, dass keine ungewollten Abhängigkeiten zwischen Layern bestehen, können die Layer in eigene Module verpackt und Abhängigkeiten über Interfaces abstrahiert werden.
Dies erhöht jedoch die interne Komplexität der Applikation.
Die Umsetzung wird aufgrund der geringen Projektgrösse deshalb nicht in unabhängigen Modulen realisiert, sondern über die Packagestruktur gelöst.
Angesichts dieser Entscheidung wird weiter darauf verzichtet, Abhängigkeiten zwischen Modulen über Interfaces zu abstrahieren.
Da es nie mehrere Implementationen einer Komponente geben wird, bringt der Einsatz von Interfaces keinen grossen Mehrwert.
Bei der Implementation wird deshalb konsequent darauf geachtet, die einzelnen Layer so zu halten, dass diese als eigenständige Module extrahiert werden können.
Für die Verwaltung der Komponenten der Serverapplikation wird folgende Packagestruktur definiert:

\begin{figure}[h]
    \centering
    \begin{minipage}[b]{0.9\textwidth}
        \dirtree{%
            .1 ch.fhnw.woweb.teamdocumentserver.
            .2 config.
            .2 domain.
            .2 persistence.
            .2 service.
            .2 web.
        }
        \caption{Packagestruktur Serverapplikation}\label{fig:packagesserver}
    \end{minipage}
\end{figure}

Der Domain Layer wird durch das Package domain abgebildet.
Dieses beinhaltet die Domänenobjekte und darf keine Abhängigkeiten auf andere Module beinhalten.
Umgekehrt dürfen alle anderen Layer Abhängigkeiten auf den Domain Layer haben.
Die Fachlogik der Applikation wird im Domain Service Layer implementiert.
Dieser wird durch das Package Service abgebildet.
Das Package Service beinhaltet alle Komponenten welche die Domänenobjekte verwalten oder den internen Zustand der Applikation führen.
Der Layer Application Services bildet die Brücke zwischen externer Infrastruktur und Domain Services.
Er wird mit den Packages persistence und web abgebildet.
Dabei definiert das Package persistence Services, welche für Interaktion mit der Datenbank verwendet werden.
Das Package web definiert die HTTP-Endpunkte, welche für die Kommunikation mit dem Frontend des Systems verwendet werden.
Letztlich beinhaltet das Package config die technische Konfiguration der Applikation.

\clearpage

\subsection{API}\label{subsec:api}

Die Backendapplikation bietet eine HTTP-Schnittstelle, welche von Frontendapplikationen verwendet werden kann.
Die Schnittstelle ermöglicht es, sich im System anzumelden, Dokumente und Änderungen zu laden und zu speichern.
Um diese Funktionalität zu ermöglichen, bietet die Schnittstelle die drei Bereiche ''Authentication'', ''Document'' und ''Message''.

\subsubsection{API Authentication}

\begin{tabbing}
Left \= Middle \= Right \= Right \kill
Beschreibung:  \> \> \> Authentifizierung mit Basic Auth\\
Endpunkt:  \> \> \> /api/v1/authentication\\
Methode \>  \> \> GET\\
Headers:  \> \>   \> Authentication: Basic \\
Response Code:  \> \>  \> 200, 401 oder 500 \\
Response Body:  \> \>  \> application/json \\
\end{tabbing}

\subsubsection{API Document}

\begin{tabbing}
    Left \= Middle \= Right \= Right \kill
    Beschreibung:  \> \> \> Dokument laden und Updates abonnieren\\
    Endpunkt:  \> \> \> api/v1/document\\
    Methode \>  \> \> GET\\
    Headers:  \> \>   \> Authentication: Basic\\
    \> \>   \> X-ClientId: UUID des Clients als text/plain\\
    Response Code:  \> \>  \> 200, 401 oder 500 \\
    Response Body:  \> \>  \> DocumentCommands als text/event-stream \\
\end{tabbing}


\subsubsection{API Message}

\begin{tabbing}
    Left \= Middle \= Right \= Right \kill
    Beschreibung:  \> \> \> Änderung an Dokument vornehmen\\
    Endpunkt:  \> \> \> /api/v1/message\\
    Methode \>  \> \> POST\\
    Headers:  \> \>   \> Authentication: Basic\\
    \> \>   \> Content-Type: application/json\\
    Body:  \> \>  \> DocumentCommand als application/json\\
    Response Code:  \> \>  \> 200, 401 oder 500 \\
\end{tabbing}

\begin{tabbing}
    Left \= Middle \= Right \= Right \kill
    Beschreibung:  \> \> \> Zuletzt gelöschten Paragraphen wiederherstellen \\
    Endpunkt:  \> \> \> /api/v1/message/restore\\
    Methode \>  \> \> POST\\
    Headers:  \> \>   \> Authentication: Basic\\
    Response Code:  \> \>  \> 200, 401 oder 500 \\
\end{tabbing}

\begin{tabbing}
    Left \= Middle \= Right \= Right \kill
    Beschreibung:  \> \> \> Dokument zurücksetzen \\
    Endpunkt:  \> \> \> /api/v1/message/reset\\
    Methode \>  \> \> DELETE\\
    Headers:  \> \>   \> Authentication: Basic\\
    Response Code:  \> \>  \> 204, 401 oder 500 \\
\end{tabbing}

\clearpage

\subsection{Komponenten}\label{subsec:komponenten}

\subsubsection{Package Domain}

Abbildung 4.3 zeigt die Klassen, des Packages Domain.
Sämtliche Klassen in diesem Package besitzen einen öffentlichen Konstruktor, welcher für alle Instanzvariablen einen Parameter entgegennimmt.
Diese Konstruktoren sind in der Abbildung nicht abgebildet.
Im Zentrum der Domäne stehen die beiden Klassen DocumentCommand und Document.

\begin{figure}[h]
    \centering
    \begin{minipage}[b]{0.8\textwidth}
        \includegraphics[width=\textwidth]{images/class-be-domain.drawio}
        \caption{Klassendiagramm Domain}
    \end{minipage}
\end{figure}

\textbf{Document}

Eine Instanz der Klasse Document repräsentiert den aktuellen Zustand eines Dokuments.
Dieser Zustand wird in der Serverapplikation geführt und verwaltet.
Ein Document besteht im Wesentlichen aus einer Liste von Paragraphs.
Diese Liste kann mutiert, aber nicht ersetzt werden.

\textbf{Paragraph}

Eine Instanz der Klasse Paragraph repräsentiert einen Abschnitt in einem Dokument.
Jeder Paragraph wird durch eine UUID identifiziert und ist einem Autor zugewiesen.
Ein Paragraph definiert weiter ein Attribut ''content'',  welches den Textinhalt des Abschnitts beinhaltet und ein Attribut ''ordinal'', welches die Position des Abschnitts im Dokument darstellt.
Letztlich hat ein Paragraph ein Optionales Attribut ''lockedBy''.
Dieses kann entweder leer (NULL) sein oder einen Autor beinhalten.
Dieses Feld kann von Consumern der API verwendet werden, um das bearbeiten eines Paragraphen zu erlauben oder verbieten.

\textbf{Author}

Eine Instanz der Klasse Author repräsentiert einen Benutzer, der an einem Dokument mitarbeitet.
Jeder Author wird durch eine UUID identifiziert und muss einen Namen definieren.
Weiter besitzt ein Author ein optionales Attribut ''image''.
Darin kann die URL zu einem Benutzerbild abgespeichert werden

\clearpage

\textbf{DocumentCommand}

Eine Instanz der Klasse DocumentCommand stellt eine Änderung am Zustand einer Document-Instanz dar.
DocumentCommands werden als einzige Entität persistiert.
Damit DocumentCommands eindeutig identifiziert werden können, beinhaltet jede Instanz ein Attribut ''id'' vom Typ UUID\@.
Diese id wird auch als Identifikator in der MongoDB verwendet.

Die Änderungen welche ein DocumentCommand darstellt, werden über die Attribute ''payload'' und ''type'' definiert.
Die Payload hat den Typ String und beinhaltet JSON-serialisierte Daten, welche die vorzunehmenden Änderung darstellen.
Das Feld ''type'' beinhaltet einen Wert aus der Enum CommandType.
Dieser Wert kann in den Serviceklassen verwendet werden, um die Payload korrekt zu deserialisieren und die nötigen Änderungen am Dokument vorzunehmen.

Das Optionale Feld ''correlationId'' kann entweder NULL oder eine UUID beinhalten.
Eine allfällige UUID zeigt immer auf die Id eines anderen DocumentCommands, welcher mit dem aktuellen Command zusammenhängt.
Dadurch wird es möglich, die Identifkation der Payload eines Commands zu verwenden, ohne die Payload deserialisieren zu müssen.

\textbf{CommandType}

Bei CommandType handelt es sich um eine Enum.
Diese Enum beinhaltet alle Arten von DocumentCommands, welche im System bekannt sind.
CommandTypes werden als ihr String Wert auf DocumentCommands persistiert.
Der verwendete CommandType bestimmt, wie ein Command verarbeitet wird.
Dies ist in Kapitel 4.5 weiter beschrieben.

\subsubsection{Package Web}

\textbf{AuthenticationController}

Die Klasse AuthenticationController implementiert einen Spring RestController.
Dieser stellt einen einzelnen GET-Endpunkt zur Verfügung, über welchen sich Benutzer mittels Basic Authentication anmelden können.

\textbf{CommandController}

Die Klasse CommandController implementiert einen Spring RestController.
Dieser Controller stellt zwei POST-Endpunkte zur Verfügung.
Über den ersten Endpoint kann eine Liste von DocumentCommands an den Server gesendet werden.
Der Endpunkt übergibt diese Liste von Commands an den DocumentService.
Diese wenden die Änderungen am Zustand des Dokuments an und leiten die Änderungen an andere Teilnehmer weiter.
Über den zweiten Endpunkt kann ein gelöschter Paragraph wiederhergestellt werden.
Die entsprechende Fachlogik wird an den DocumentService delegiert.
Letztlich stellt der Controller einen DELETE-Endpunkt zur Verfügung.
Über diesen kann der Zustand des Dokuments zurückgesetzt werden.

\textbf{DocumentStreamUpdateController}

Die Klasse DocumentStreamUpdateController implementiert einen Spring RestController.
Dieser Controller ermöglicht es, den aktuellen Zustand eines Dokumentes zu laden und Änderungen am Dokument zu abonnieren.
Der Endpunkt, welcher dazu zur Verfügung steht erwartet, dass der Custom Header ''X-ClientId'' gesetzt ist.
Dieser muss einen die Id des Authors, der die Daten laden möchte, beinhalten.
Das Laden des Dokuments und das Erstellen der Abonnierung wird an den DocumentService delegiert.
Als Rückgabetyp wird ''Flux$<$DocumentCommand$>$'' verwendet.
Dadurch ist es möglich den Status des Documents und alle folgenden Änderungen in einem Stream zurückzugeben.

\clearpage

\subsubsection{Package Services}

\begin{figure}[h]
    \centering
    \begin{minipage}[b]{0.8\textwidth}
        \includegraphics[width=\textwidth]{images/class-be.drawio}
        \caption{Klassendiagramm Services}
    \end{minipage}
\end{figure}


\textbf{DocumentService}

Die Klasse DocumentService ist dafür verantwortlich, erhaltene Anfragen für Dokumente und Änderungen an Dokumenten zu verarbeiten.
Sie delegiert die entsprechende Fachlogik an die Klassen ActiveSessionService, DocumentProcessor und DocumentCommandRepository.

Die Methode process erlaubt es, eine Liste von DocumentCommands zu verarbeiten.
Die Methode subscribe erlaubt es, einen Stream des aktuellen Zustands des Dokuments und aller künftigen Änderungen an einem Dokument anzufragen.
Beide Abläufe werden in Kapitel 4.4 beschrieben.

\textbf{DocumentProcessor}

Die Klasse DocumentProcessor führt den Zustand des Dokuments, welches mit der Applikation verwaltet wird.
Er ist dafür Verantwortlich, Änderungen an diesem Dokument vorzunehmen.
Dazu besitzt Sie ein privates Attribut document vom gleichnamigen Typ.
Über die Methode process kann ein einzelner DocumentCommand angewendet werden.
Der Processor verarbeitet den Command anhand des gesetzten CommandTypes.
Dabei muss er allfällige Konflikte erkennen und auflösen.
Nach der Verarbeitung des Commands werden alle Änderungen und Konfliktlösungen als DocumentCommands zurückgegeben.
\clearpage
\textbf{ActiveSessionService}

Die Klasse ActiveSessionService führt den Zustand der aktiven Nutzer einer Session.
Dazu führt die Klasse eine Liste der Identifikatoren aller aktiven Nutzer.
Der Service bietet Methoden um die aktiven Benutzer auszulesen, einen neuen Benutzer zu registrieren und einen Benutzer zu entfernen.

\subsubsection{Package Persistence}

\textbf{DocumentCommandRepository}

Das Interface DocumentCommandRepository erweitert das Interface ReactiveCrudRepository von Spring.
Es kann damit verwendet werden um Create, Read, Update und Delete Optionen für DocumentCommands in der angebundenen MongoDB auszuführen.

\subsubsection{Konfiguration}

\textbf{Spring}

Alle Klassen im Package "Service" sind mit der Spring-Boot-Annotation ''@Service'' versehen.
Sie können damit automatisch von Spring-Boot instanziiert werden und stehen Sie in Spring Beans zur Verfügung und können über Constructor-Injection verwendet werden.

Alle Klassen im Package "web" sind mit der Spring-Boot-Annotation ''@RestController'' versehen.
Sie können damit automatisch von Spring-Boot instanziiert werden.

\textbf{application.yml}

Die Datei application.yml beinhaltet die konfigurierbaren Werte der Serverapplikation.
Dies beinhaltet die Konfiguration der angebundenen MongoDB, Referenzen zu Umgebungsvariablen mit User Credentials und Logging Konfiguration.

\textbf{TeamDocumentServerProperties}

Die Konfigurationsklasse TeamDocumentServerProperties ist mit der Annotation ''@ConfigurationProperties(prefix = "teamdocument")'' versehen.
Sie kann in den Serviceklassen verwendet werden, um auf Werte aus dem application.yaml zuzugreifen.

\textbf{WebConfig}

Die Klasse WebConfig beinhaltet die Konfiguration für SpringSecurity.

\clearpage

\subsection{Abläufe}

\subsubsection{Dokument laden und Änderungen abonnieren}

\begin{figure}[h]
    \centering
    \begin{minipage}[b]{1\textwidth}
        \includegraphics[width=\textwidth]{images/seq_init_subscription.drawio}
        \caption{Sequenzdiagramm Document Subscription}
    \end{minipage}\label{fig:figureseqsub}
\end{figure}

Die Komponente DocumentUpdateStreamController erlaubt es, ein Dokument zu laden und Änderungen an diesem Dokument zu abonnieren.
Dazu bietet der RestController einen Endpunkt, welcher den ein Resultat vom Typ Flux$<$DocumentCommand$>$ zurückgibt.
Nachdem eine Anfrage beim Controller eingegangen ist, delegiert dieser die Erstellung des Flux an den DocumentService.
Dieser registriert den Client hinter der Subscription beim ActiveSessionService und trägt anschliessend Informationen aus drei Quellen zusammen.
Zuerst wird der aktuelle Stand des Dokuments beim DocumentProcessor angefragt.
Anschliessend wird eine Liste aller aktiven Clients des Dokuments aus dem ActiveSessionService geladen.
Letztlich wird eine Subscription für Änderungen am Dokument erstellt.
Der DocumentService führt dazu eine Instanzvariable vom Typ Sink.
Alle Änderungen werden nach der Anwendung in diesen Sink geschrieben.
Eine Subscription auf diesem Sink beinhaltet damit alle Änderungen, welche vorgenommen wurden.
Die Informationen zum initialen Stand des Dokuments, aktiven Clients und die Subscription auf Änderungen werden in einem einzelnen Flux zusammengefasst und zurückgegeben.
Nachdem der Flux geschlossen wird, wird die Registrierung des aktiven Benutzers im ActiveSessionService wieder entfernt.

\subsubsection{Änderungen verarbeiten}

Die Komponente CommandController erlaubt es, Änderungen an einem Dokument vorzunehmen und zu persistieren.
Der Controller bietet dazu einen HTTP-Endpunkt, über welchen ein JSON-serialisierter DocumentCommand übergeben werden kann.
Ein DocumentCommand definiert unter anderem einen Typ, welcher ein Wert der Enum CommandType sein muss und eine Payload vom Typ String.
Die Payload ist wiederum ein JSON-serialisiertes Objekt.
Die Verarbeitung eines DokumentCommands wird vom Controller an den DocumentService delegiert.
Dieser übergibt den Command wiederum an den DocumentProcessor.
Dieser wendet Änderungen an und löst Konflikte auf.
Die übergebenen Commands und allfällige Commands zur Konfliktlösung werden vom DocumentProcessor zurückgegeben.
Der DocumentService übergibt diese an das DocumentCommandRepository zur Persistierung in der Datenbank.
Anschliessend werden die persistierten Commands über die Instanzvariable vom Typ Sink veröffentlicht.
Registrierte Clients haben eine Subscription auf diesen Sink und werden so über die Änderungen informiert.

\begin{figure}[h]
    \centering
    \begin{minipage}[b]{1\textwidth}
        \includegraphics[width=\textwidth]{images/seq_send_command.drawio}
        \caption{Sequenzdiagramm Command verarbeiten}
    \end{minipage}\label{fig:figuresqcmd}
\end{figure}

\subsubsection{Fehlerbehandlung}

Fachliche Fehler, die während der Verarbeitung von Änderungen an einem Dokument auftreten, werden innerhalb des DocumentProcessors behandelt.
Dabei werden immer zusätzliche Änderungen generiert, welche Konflikte lösen.
Die Verarbeitung wird nie durch das Werfen von Exceptions unterbrochen.
Es ist möglich, dass die Verarbeitung einer Änderung oder das Veröffentlichen von Änderungen wegen einem technischen Fehler fehlschlägt.

Wenn eine Exception beim Verarbeiten einer Änderung auftritt, wird die Verarbeitung abgebrochen und die Exception weiter geworfen.
Exceptions werden in den Controller Klassen abgefangen.
Im Fehlerfall wird dort die Exception in das Log geschrieben und eine ResponseEntity mit Statuscode 500 zurückgegeben.

Wenn eine Exception beim Veröffentlichen einer Änderung auftritt, wird die Subscription geschlossen.
Dadurch wird die Verbindung des betroffenen Clients getrennt.
Es liegt in der Verantwortung des Clients die Verbindung erneut zu öffnen und den aktuellen Stand des Dokuments zu laden.

\clearpage

\subsection{Zustands- und Konfliktmanagement}\label{subsec:statemgmt}

Um sicherzustellen, dass ein Dokument auf allen Clients einen konsistenten Zustand hat, wird der Zustand des Dokuments in der Serverapplikation geführt.
Änderungen von Clients müssen an den Server gesendet werden.
Dieser wendet die Änderungen auf dem Dokument an und leitet die Änderungen an alle anderen Clients weiter.
Während der Anwendung von Änderungen, ist es in der Verantwortung der Serverapplikation, sicherzustellen, dass das Dokument in einem Konsistenten Zustand bleibt.
Wenn nötig, erstellt es dazu zusätzliche Änderungen und veröffentlicht diese ebenfalls an alle Clients.

Die entsprechende Fachlogik ist in den Klassen DocumentService und DocumentProcessor implementiert.
Dabei ist die Klasse DocumentProcessor für die Zustandsverwaltung des Dokuments verantwortlich.


\subsubsection{Grundsatz}

Der DocumentProcessor hat zwei private Instanzvariablen, die für das Zustandsmanagement relevant sind.
Die Variable \textbf{document} ist vom gleichnamigen Typ.
Diese Document-Instanz stellt die ''Source Of Truth'' für den Zustand des Dokuments dar.
Ein Document besteht im Wesentlichen aus einer Liste von Paragraphen.
Diese Liste wird mit ''synchronizedList(mutableListOf<Paragraph>())'' initialisiert.
Dadurch wird sichergestellt, dass der Zugriff für die Liste zwischen der Verarbeitung einzelner Commands synchronisiert ist.
Die Variable \textbf{lock} hat den Typ ReentrantLock.
Die Verarbeitung einiger DocumentCommands darf nicht parallel passieren, weil dadurch Konflikte ausgelöst werden können.
Die Verarbeitung solcher DocumentCommands wird durch die Verwendung dieses ReentrantLocks gesperrt.

\subsubsection{Zustandsänderungen}

Alle Änderungen werden als Instanzen der Klasse DocumentCommand an den Server übermittelt.
Nachfolgend wird beschrieben, welche Arten von DocumentCommands unterstützt sind, wie diese verarbeitet werden und wie mit möglichen Konflikten umgegangen wird.

Ein DocumentCommand mit Typ \textbf{INITIAL} stellt den vollständigen Zustand eines Dokuments dar.
Er beinhaltet als Payload eine Liste der Paragraphen in diesem Dokument.
INITIAL Commands werden im Backend generiert, um den vollständigen Zustand eines Dokuments zu veröffentlichen.
Eingehende INITIAL Commands werden im DocumentProcessor nicht angewendet.

Ein DocumentCommand mit Typ \textbf{ADD\_PARAGRAPH} fügt einen neuen Abschnitt zum Dokument hinzu.
Er beinhaltet als Payload einen einzelnen Paragraphen.
Dieser Paragraph wird im DocumentProcessor deserialisiert und der Liste von Paragraphen im Dokument hinzugefügt.
Es ist möglich, gleichzeitig zwei Commands eingehen, welche an derselben Stelle im Dokument einen Paragraphen einfügen möchten.
Deshalb wird hier sichergestellt, dass die Ordinalnummern aller Paragraphen korrekt sind und keine Nummer doppelt vorkommt.
Während diese Korrektur vorgenommen wird, dürfen keine anderen Änderungen möglich sein.
Deshalb wird die Verarbeitung eines ADD\_PARAGRAPH Commands mit dem ReentrantLock versehen.
Nach der Verarbeitung werden die ADD\_PARAGRAPH und UPDATE\_PARAGRAPH\_ORDINALS Commands zurückgegeben, damit Sie veröffentlicht werden können.

Ein DocumentCommand mit Typ \textbf{REMOVE\_PARAGRAPH} entfernt einen Paragraphen aus dem Dokument.
Ein beinhaltet als Payload die UUID, des zu entfernenden Abschnitts.
Bei der Verarbeitung wird der Abschnitt mit der gegebenen Id entfernt.
Ist der Abschnitt bereits entfernt, wird kein Fehler geworfen.
Nachdem ein Paragraph entfernt wurde, muss sichergestellt werden, dass es keine Lücke in den Ordinal Nummern der Paragraphen gibt.
Deshalb wird auch hier sichergestellt, dass die Ordinalnummern aller Paragraphen korrekt sind.
Anschliessend werden die REMOVE\_PARAGRAPH und UPDATE\_PARAGRAPH\_ORDINALS zurückgegeben.
Um sicherzustellen, dass die Ordinalnummern korrekt gesetzt werden, ist auch diese Verarbeitung mit dem ReentrantLock abgeschlossen.

Ein DocumentCommand mit Typ \textbf{UPDATE\_PARAGRAPH} aktualisiert den Textinhalt eines Abschnitts.
Er beinhaltet als Payload einen einzelnen Paragraphen.
Bei der Verarbeitung wird der relevante Abschnitt im Dokument gefunden und dessen Inhalt überschrieben.
Hier wird bewusst kein explizites Konfliktmanagement betrieben.
Paragraphen dürfen nur bearbeitet werden, wenn der Benutzer den Paragraph für sich gesperrt hat.
Da ein Paragraph immer nur von einem Benutzer gesperrt sein kann und Updates in derselben Reihenfolge wie sie geschehen eingehen, können diese Updates immer angewendet werden.
Es ist möglich, dass sich das Sperren eines Pargraphen von zwei Benutzern überschneidet.
Am Ende darf aber immer nur ein Benutzer den Paragraphen gesperrt haben.
In diesem Fall ist es das gewünschte Verhalten, dass die Änderungen dieses Benutzers alle anderen Änderungen am selben Abschnitt überschreiben. 

Ein DocumentCommand mit Typ \textbf{UPDATE\_PARAGRAPH\_ORDINALS} aktualisiert die Ordinalnummern von Abschnitten.
Er beinhaltet als Payload eine Liste von Paragraphen.
Bei der Verarbeitung dieses Commands werden die Ordinalnummern aller Abschnitte mit den Ordinalnummern aus der Payload überschrieben.
Anschliessend wird sichergestellt, dass die Ordinalnummern aller Paragraphen korrekt sind und keine Nummer doppelt vorkommt.
Es werden darauf der erhaltenen Command und Commands zur Konfliktlösung zurückgegeben.
Um sicherzustellen, dass die Ordinalnummern korrekt gesetzt werden, ist auch diese Verarbeitung mit dem ReentrantLock abgeschlossen.

Ein DocumentCommand mit Typ \textbf{UPDATE\_AUTHOR} aktualisiert den Namen eines Authors, der das Dokument bearbeitet.
Er beinhaltet als Payload eine Author-Instanz.
Der Name dieses Authors wird auf allen Abschnitten im Dokument aktualisiert.
Anschliessend wird der Command zurückgegeben.
Es wird hier kein explizites Konfliktmanagement betrieben, da ein Benutzer immer nur an genau einem Gerät arbeiten kann.
Sollte derselbe Benutzer auf mehreren Clients verwendet werden und gleichzeitig den Namen ändern, können sich diese Änderungen überschreiben.
In diesem Fall wird das zuletzt gesendete Update angewendet und veröffentlicht.
Damit ist der Zustand auch bei Konflikten konsistent.

Ein DocumentCommand mit Typ \textbf{UPDATE\_LOCK} erlaubt es einen Abschnitt durch einen Benutzer zu ent-/sperren.
Dies wird in den Clients verwendet, um sicherzustellen, dass nur ein Benutzer gleichzeitig an einem Paragraph arbeiten kann.
Die Payload dieses Commands beinhaltet als Payload den Pargraphen, der gesperrt werden soll.
Dieser Paragraph wird in der Liste von Pargraphen gefunden und durch setzten des lockedBy Attributs gesperrt.
Ist auf der Payload kein lockedBy Attribut gesetzt, wird die Sperre entfernt.
Dabei wird sichergestellt, dass eine Sperre nur durch den Benutzer, der sie erstellt hat entfernt werden kann.
Versucht ein anderer Benutzer, die Sperre aufzuheben, wird die Verarbeitung abgebrochen und ein Command welcher die Sperre zurücksetzt zurückgegeben.
Damit sich das Sperren von Paragraphen zwischen Benutzern nicht überschneiden kann, ist die Verarbeitung dieses Commands mit dem ReentrantLock versehen. 

Ein DocumentCommand mit Typ \textbf{ADD\_CLIENTS} teilt mit, dass ein neuer Bearbeiter am Dokument existiert.
Dieser Command führt zu keiner Änderung am Dokument und wird im DocumentProcessor nicht verarbeitet.
Er wird bei der Registrierung einer Subscription erstellt und veröffentlicht.

Ein DocumentCommand mit Typ \textbf{REMOVE\_CLIENT} teilt mit, dass die Verbindung eines Bearbeiters getrennt wurde.
Bei der Verarbeitung dieses Commands werden alle Paragraphen, welche durch diesen Bearbeiter gesperrt waren entsperrt.
Dadurch wird sichergestellt, dass Paragraphen nicht gesperrt sind wenn ein Benutzer den Client beendet oder die Verbindung abbricht.
Der Command wird durch den ActiveSessionService generiert und vom DocumentProcessor verarbeitet, nachdem ein Client die Verbindung getrennt hat.

\clearpage
