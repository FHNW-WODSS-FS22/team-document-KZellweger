\section{Testing}
%TODO: Lead @Peter
\subsection{Frontend}
%TODO: @Peter

\subsection{Backend}

\subsubsection{Unit Tests}

Sämtliche Service- und Controller Klassen werden mit Unit-Tests getestet.
Dazu wird das Framework JUnit verwendet.
Die Unit Tests testen jeweils genau eine Klasse.
Sämtliche Abhängigkeiten auf andere Services werden mit dem Framework Mockito gemocked.

\subsubsection{Lasttests}

Neben den einfachen Unit Tests wurden für den DocumentProcessor Lasttests implementiert.
Diese stellen sicher, dass der DocumentProcessor Änderungen auch unter grössere Last schnell verarbeitet und dabei einen konsistenten Zustand im Dokument erstellt.
In diesen Lasttests werden drei Benutzer simuliert welche parallel je 512 DocumentCommands verarbeiten lassen.
Dabei wird vor jedem Verarbeitungsschritt eine zufällige Verzögerung von max. einer Sekunde eingebaut.
Der Test stellt sicher, dass im Schnitt nicht mehr als vier Millisekunden für die Verarbeitung eines Commands verwendet.
Der Test prüft weiter, dass das Dokument nach der Verarbeitung den erwarteten Zustand hat.

\subsubsection{Application Tests}

Mit der Testklasse TeamDocumentServerApplicationTests wird die Backendapplikation als Ganzes getestet.
Dazu wird der gesamte Application Context hochgefahren.
Anschliessend werden DocumentCommands direkt über die Controller Klassen verarbeitet.
Dabei wird geprüft, dass Änderungen korrekt angewendet und veröffentlicht werden.

In den Application Tests werden zudem einfache Lasttests ausgeführt.
Gleich wie bei den Lasttests auf dem DocumentProcessor werden hier drei Benutzer simuliert welche parallel je 512 DocumentCommands verarbeiten lassen.
Anschliessend wird sichergestellt, dass im Schnitt nicht mehr als 40 Millisekunden für die Verarbeitung eines Commands verwendet wird.
Es wird weiter geprüft, dass das Dokument nach der Verarbeitung in einem Konsistenten zustand ist.



\subsection{End to End Test}
%TODO: @Peter
