\section{Fazit}

Die umgesetzte Lösung erlaubt es ein Dokument parallel auf einem öffentlich erreichbaren Server zu bearbeiten.
Die funktionalen Anforderungen an die Applikation sind grösstenteils umgesetzt.
Das System hat aber noch diverse Lücken in den Bereichen Konfliktlösung und Zustandsverwaltung.
In gewissen Situationen ist es möglich, dass inkonsistente Ordinalnummern entstehen und Änderungen verloren gehen.

Der Ansatz im Backend mit einem reaktiven Stack zu arbeiten und Updates über die Eventsource zu versenden funktioniert sehr gut.
Neue Funktionalität könnte elegant mit zusätzlichen Command-Types hinzugefügt und in die Processor Chain eingebunden werden.
Der Processor erreichte im isolierten Lasttest eine durchschnittliche Verarbeitungszeit von 4 ms für einen einzelnen Command.
Der Application Test von Controller bis zur Eventsource brauchte im Schnitt 40 ms für die komplette Verarbeitung.

Analog scheint auch der Ansatz im Frontend mit React und Redux gut zu funktionieren.
Die Verarbeitung der Updates von der Eventsource lässt sich mit dem gleichen Prinzip wie im Backend einfach erweitern.
Ein neuer Command entspräche dann einer neuen Funktion im Reducer.
Die Redux Komponenten kommen ohne eigene Logik aus, was die gesamte Frontendapplikation übersichtlich und einfach zu warten macht.
Die Implementation der Reducer-Funktionen hat sich als herausfordernder als erwartet dargestellt.
So ist es im Endeffekt nicht gelungen, Reducer zu implementieren, welche einen garantiert Konsistenten Zustand herstellen.

Was sich retrospektiv als Fehler erwiesen hat, war die Entscheidung die Commands aus dem Frontend via HTTP an den Server zu senden.
Für jedes Update wird eine TCP Verbindung aufgebaut und entsprechend auch gewartet bis diese wieder abgebaut ist.
Das führt bei schnellen aufeinanderfolgenden Updates über das Internet zu einem Request-Stau.
Treffen am Server mehr Requests ein als dieser verarbeiten kann ist auch die Reihenfolge der eingehenden Commands nicht mehr garantiert.
Diese Probleme wurden erst nach erfolgreichem Deployment auf dem öffentlichen Server sichtbar.
Aufgrund des schon fortgeschrittenen Projektstandes haben wir dieses Grundkonzept nicht gänzlich überarbeitet, sondern versucht das Netzwerk mit einer debounce funktion zu entlasten.
Die Konfliktbehandlung des Backends geht davon aus, dass Updates in der korrekten Reihenfolge (FIFO) ankommen.
Da es dennoch noch vorkommen kann, dass sich einzelne Requests überholen und die eigenen Updates vom Client direkt im Redux Store angewendet werden, können inkonsistente Zustände im Frontend entstehen.
Dieses Problem ist die zweite Ursache für die bekannten Lücken im Bereich Konfliktlösung und Zustandsverwaltung.

Am Ende dieses Projektes würden wir die Architektur des Systems grösstenteils beibehalten.
Dabei allerdings das Versenden von Commands von Backen und Clients immer über denselben Kanal gehen lassen.
