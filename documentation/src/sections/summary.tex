\section{Ausgangslage}

''Im Modul Workshop Distributed Software Systems (DSS) und Workshop Web (WOWEB) werden die Themen der Vertiefungsrichtung Verteilte Software Systeme und der Vertiefungsrichtung Web in einem durchgehenden Beispiel angewendet und umgesetzt.''\footnote{Zitat Aufgabenstellung: \href{https://github.com/FHNW-WODSS-FS22/team-document-KZellweger/blob/master/readme.md}{https://github.com/FHNW-WODSS-FS22/team-document-KZellweger/blob/master/readme.md}}
Im Rahmen dieses Moduls soll eine ''light'' Version von Google Docs umgesetzt werden.
Diese Lösung soll kollaboratives Arbeiten an einem einzelnen Dokument erlauben.

Eine zentrale Anforderung an das System ist die konsistente und verzögerungsfreie Darstellung des Dokuments auf mehreren Clients.
Um diese Herausforderung zu lösen, müssen Applikationsprotokolle und Konzepte definiert werden, welche es erlaubt den Zustand eines Dokumentes zu verwalten und allfällige Konflikte zu lösen.
Es müssen Technologien und Kommunikationsprotokolle gewählt werden, welche es ermöglichen diese Konzepte umzusetzen.

In den folgenden Kapiteln werden die erarbeiteten Konzepte und deren Implementierung vorgestellt.
Dabei werden die gewählten Technologien und die umgesetzten Komponenten beschrieben.
